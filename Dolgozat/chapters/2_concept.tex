\Chapter{Fejlesztői környezet}

Az OpenTTD már a kezdetlegi fázisaitől lehetőve tette azt, hogy a felhasználók saját AI-t fejleszthessen a játékhoz. Ennek a helyzetnek az előteremtésére a játék fejlesztői csapata létrehoztak egy API-t amit a Squirrel nyelv használatával elérést ad a játék back end-jéhez. Mivel a játék forrása tartalmaz egy fordítót, így a felhasználónak csak egy szövegszerkeztőre van szüksége az AI megteremtéséhez.



\Section{Squirrel}

Az OpenTTD-hez készült kiegészítő szkriptek, AI-ok, módosítások nyelve a Squirrel. A Squirrel egy egyszerű szkriptnyelv aminek a szintaktikája sokban hasonlít a C++-ra. Legfőképpen két oka volt annak, hogy ezt a nyelvet választották. Elsősorban, sokban hasonlít a C++-ra, ami a fő nyelve a játék teljes forráskódjának. Másodsorban pedig, a Squirrel egy magas szintű,  imperatív, objektum orientált programozási nyelv, ami olyan célokkal lett megalkotva mint, egyszerű szkriptelés, alacsony memória használat és képesség a valós idejű programokkal való működésre, mint amilyenek a játékok is.

Ahogy korábban ez említve volt, a szintakszis főképpen C++-ra vagy C/Java-ra hasonlít, azonban a teljesítménye és és implementációja inkább tükrözi a Python-t, JavaScript-et és Lua-t. A Lua több helyen is használt program, mint például az Adobe Photoshop Lightroom, MySQL Workbench, VLC media player és még sok más. Játékokban való alkalmazására példa a Far Cry és a Civilization V.

A Lua-hoz képest a Squirrel kevésbé elterjedt nyelv, és kevesebb eredményt is tud felmutatni, azonban néhány példa a használatára ismertebb játékokban például Left 4 Dead 2, Portal 2, GTA IV Multiplayer Mod és Counter Strike: Global Offensive.

A Squirrel főbb tulajdonságai:

\begin{itemize}
\item delegáció
\item osztályok és öröklődés
\item magasabb rendű függvények
\item érvényességi terület (scope)
\item generátorok
\item kivételkezelés
\item rekurzió
\item automata memória kezelés
\end{itemize}

Ezen felsorolt funkciói miatt a Squirrel egy nagyon rugalmas és egyszerűen implementálható eszköz, ami tökéletes egy játékhoz való API elkészítéséhez. További előnyök ami miatt az OpenTTD fejszeltői a Squirrel mellett döntöttek a C++ helyett az AI támogatáshoz:

\begin{itemize}
	\item Beépített multi-platform támogatás
	\item A C++-ban írt AI-ok teszteléséhez az egész játék kódjának az újbóli összeállítására lett volna szükség
	\item C++-ban a hibák az egész játék összeomlásához vezetnek, de Squirrel-ben a hibák a játékban futtatott virtuális gépen belül maradnak, ami legrosszabb esetben is csak az AI leállásához vezethet
\end{itemize}

Érdemes megemlíteni, hogy a C++-hoz képest a Squirrel sebessége rosszabb, azonban ebben az alkalmazás módban kevésbé észrevehető. A legfőbb ok amiért a játék szempontjából hasznos a Squirrel használata az a biztonság és a hordozhatóság. Mindez ahhoz vezetett, hogy a játékhoz készült NoAI Keretrendszer, amiről a későbbiekben részletesen is beszélünk, is a Squirrel nyelvben lett megírva, ami támogatja az AI-ok működését.

Ahhoz hogy Squirrel programot készítsünk, csak a Squirrel-re, és egy szövegszerkesztőre van szükségünk. Az előbbi a Squirrel weboldaláról letölthető. Az utóbbi pedig bármilyen fejlesztői környezet, vagy egyszerű szövegszerkesztő lehet. Az AI fejlesztése alatt a Visual Studio Code-ot használtam Squirrel Language Supports kiegészítővel, ami hozzáad szövegkiemelést, néhány alapvető automata kitöltést és formázási lehetőségeket.


\Section{NoAI Keretrendszer}

A korábban fejlesztett és jelenleg támogatott AI-ok mind a NoAI Framework nevű keretrendszerre épülnek. Egy másik lehetőség egy AI megvalósítássára a játékon belül a játék forráskódjának szerkeztése és annak újrafordítása lenne, azonban a keretrendszer által adott egyszerűsítések jelentősen a háttérbe szorítják ezt az opciót, amellett hogy ez sokkal bonyolultabbá tenné az AI implementálását, az elérhetőségét is korlátozná.

Ahogyan az korábban említve volt, az OpenTTD játék forráskódja nyílt, és a hozzá tartozó, felhasználók által készített szkriptek nagyrésze is az. A legtöbb az Általános Nyilvános Licenc (GNU) 2.0-ás vagy 3.0 változata alatt érhetőek el. Emiatt a fejlesztés megkezdése előtt két lehetőségünk van, az egyik egy AI megvalósítása teljesen a semmiből, vagy pedig egy már létező AI kódjának az átformázása a mi céljainkra, vagy esetleg már létező kódrészletek használata.

A keretrendszer részeként érdemes megemlíteni, a játékhoz elérhető előre készített könyvtárakat. Néhány egyszerű, de fontos elemet tartalmaznak, amelyek közül a saját megvalósítás alatt használunk is. Ilyenek például különböző útvonalkereső algoritmusok, amelyek út vagy vasútvonalakat tudnak létrehozni adott pontok között. A fejlesztő nincs ezeknek az elemeknek a használatára kötelezve, saját implementációt is készíthet egy útvonalkereső algoritmushoz.

\Section{Minimális AI}

Először is, hogyan tudunk megvalósítani egy AI-t a játék keretein belül. Három dologra van szükségünk ahhoz hogy ezt létrehozzuk. Először is a játék mappáján belül található "ai" nevű mappában kell létrehoznunk egy mappát, szbvány szerűen az AI nevével. Aztán ebben a mappában 2 fájlra lesz szükségünk, mindkettő Squirrel nyelvű ".nut" kiterjesztésűnek kell lennie, az egyik "info.nut" a másik pedig "main.nut" néven kell létrehoznunk.

AZ "info.nut" fájlban tudjuk megadni az általánmos információkat az AI-unkról, például a szerző nevét, az AI nevét, a játékban elérhető rövid leírását, a kód verziószámát és kiadási dátumát. Ebben az osztályban van lehetőségünk különböző beállítások definiálására is, amihez a felhasználó az egyik a játékban található menüpontban tud hozzáférni. Ahhoz hogy a játék magjának is a tudtára adjuk hogy ez egy AI a RegisterAI() nevű függvényt kell meghívnunk.

Ha ezzel készen vagyunk, a másik "main.nut" fájlban tudjuk az AI viselkedését meghatározni. Itt kell megadnunk a fő ciklusát az AI-nak valamint egyéb osztályok és forrásfájlokra való hivatkozásokat. Az osztálynak néhány alapértelmezett funkciója van melyek a következőek:

\begin{itemize}
	\item Start() - Itt kapja meg a vezérlést az AI
	\item Save() - Ezzel tudjuk lementeni az AI állapotát amikor a játékmenet mentése történik
	\item Load() - A játék betöltésénél itt tudjuk betölteni a korábban lementett állapotát az AI-nak
\end{itemize}

Ezek közül a függvények közül számunkra a Start() a legfontosabb, ugyanis, amikor elindítjuk az AI-t a játékon belül, ezt a függvény fog meghívódni, és az itt megírt programrészek fognak lefutni. Éppen emiatt, ha azt szeretnénk hogy a játék közben folyamatosan fusson az AI-unk, egy végtelen ciklusba érdemes belerakni a főbb parancsokat, amelyek ismétlődésével jön létre az AI játékmenete.

Ahhoz, hogy ez az AI megjelenjen játék közben, még a főmenüben az "MI/Játékszkript beállítások" menüpont alatt be tudjuk állítani hány ellenfelet szeretnénk a játékunkban. Valamint azt is rögzíteni tudjuk hogy az adott ellenfél melyik feltelepített AI legyen. Alapértelmezetten ez a beállítás véletlenszerű. Ha ezt beállítottuk, a játék indításakor a mi AI-unknak is meg kell jellennie a cégek között.


\Section{NoAI API}

Ahogyan az korábban említve volt a játék egy egészen bőkezű API-val rendelkezik, aminek a segítségével elérhetőek lesznek a fejlesztő számára azok a funkciók és információk mint ami egy felhasználó számára is elérhetőek a játék közben. Értve itt például a pálya állapotának lekérdezését, vagy utak építését. Minden funkció különböző osztályokba vannak rendezve aszerint hogy milyen feladatot látnak el. Például az AIStation osztályban találunk olyan függvényeket amelyekkel egy állomáshoz kapcsolódó információkat tudunk lekérdezni, vagy módosítani.

A különböző funkciók mellett az API felel a különböző AI-ok ütemezéséért is a játék futása alatt. Ugyanis a játék futása közben, egy vagy több AI-nak a játék időközönként futási időt ad, amikor a megírt programkód ténylegesen lefut. Ez a játékban egy round-robin módszerű ütemezővel van megoldva, vagyis a vezérlés először a játéknál van, aztán több AI esetén, először az egyik kap időt parancsok végrehajtására, aztán pedig egy másik, amíg mindegyik sorra nem jut, aztán a folyamat előlről kezdődik, így biztosítva hogy a futási idő egyenletesen legyen elosztva. 

Az AIController nevű osztály amivel már a minimális AI elkészítése során is találkozhatunk, ugyanis, a "main.nut"-on belül található osztályunknak az API ezen osztályát kell kiterjesztenie ahhoz, hogy a Start() függvényt meg tudjuk hívni az AI indításakor. Valamint ezen keresztül kérdezhetőek le információk a játék jelenlegi állapotáról.

A fejlesztő szempontjából hasznosak lehetnek az AIError, AIEvent és AILog osztályok. Ezek segítségével van lehetőség kommunikálni a játék konzolába a program futásából különböző debugoláshoz hasznos üzeneteket, vagy az AIError segítségével észlelhetőek és lekérdezhetőek a hibák.

Az AI szempontjából több hasznos osztály is rendelkezésre áll, egyrészt az AICompany és AIAccounting amelyek a céggel kapcsolatos információk lekérdezésében és változatásában és a cég pénzügyeinek a kezelésében segítenek.

De talán a legfontosabbak a megfelelő működésért az AIMap és AITile osztályok a leginkább felelősek. Ugyanis a játék különböző infrastruktúrák kiépítésére alapszik. Ahhoz pedig hogy megfelelő infrastruktúrát tudjunk építeni először is alaposan fel kell mérnünk a játék pályáját, amit az AIMap függvényeinek segítségével tehetünk meg. Amikor a térkép egy részét lekérdezzük vagy vizsgáljuk, eredményül AITile osztályú objektumokat kapunk vissza, amik az egyes mezőket jelölik amik a térképen megtalálhatóak. Ezeknek a mezőknek van számtalan tulajdonságuk amelyek megszabják hogy hogyan jelenik meg a felhasználó számára egy adott mező. Például, van e út a mezőn vagy a mező esetleg víz-e? Az AIMap rendelkezik még több olyan függvénnyel amelyek a későbbiekben hasznosak lesznek az útvonalak tervezésénél. Ilyen például a DistanceManhattan(TileIndex tilefrom, TileIndex tileto) aminek a segítségével két mező közötti manhattan távolságot tudjuk lekérdezni. Ezzel tudjuk adott mezők közötti legrövidebb távolságot is kiszámítani.

Ezen felül a különböző szállításó módokhoz kapcsolódóan is vannak megfelelő osztályok, mint AIRoad, AIRail. Az AIRoad osztály ad rendelkezésünkre minden olyan függvényt amelyekre az utakkal kapcsolatban szükségünk lehet, például utak építése, garázsok építése vagy különböző információk lekérdezése a mezőn lévő útról.

Az API részeként elérhető az AIList osztály is, ami egy láncolt lista szerűen képes tárolni több típusú elemet is. Ennek az osztálynak a függvényei kifejezetten hasznosnak bizonyultak a fejlesztés során, mivel ennek az osztálynak a további gyerekosztályai például az AITownList és AITileList gyakran használatba kerültek. Ezek olyan listák amelyek adott esetben a pályán található városokat, vagy mezőket, attól függően milyen listáról beszélünk, tartalmazzák. Az adott elemekhez a listán belül tudunk értéket is rendelni és bár ez csak integerekre van korlátozva így is sok helyen felhasználható, főleg a Valuate() használatával, amivel az egész listának egyszerre adhatunk értéket egy értékelőfüggvény segítségével. Így például egyszerűen eldönthetjük hogy a listában szereplő mezők közül melyek utak, mivel attól függve hogy út e vagy sem, a függvény ad az elemeknek egy 0 vagy 1 értéket.

\Section{Nyomkövetés}

Az AI tesztelésére a játékon belül van lehetőségünk. Minden alkalommal amikor elindítunk egy játékot, és legenerálódott a világ, valamint beállítottuk hogy rivális cégként megjelenjen az AI-unk is, akkor minden alkalommal előről indul az AI-unk. Az AI által generált hibák követésére két módszerünk van, az egyik a játék konzolja, amit a "0" gomb lenyomásával tudunk elérni egy játékon belül. itt tudunk parancsokat kiadni amivel elindíthatjuk manuálisan is az AI-t, vagy leállíthatjuk. A másik opció pedig az "MI/Játékszkript nyomonkövetés" amit a játék közben a felső eszköztár utolsó menüpontjában találunk. Ebben az ablakban jelennek meg a szövegek, amiket az AILog segítségével kiíratunk, így nyomon követhető hogy éppen mi is történik az AI-al. Emelett található itt egy gomb, amivel újra tudjuk tölteni az AI-t, és egy gombot amivel ugyan azt a menüt tudjuk a játék közben is elérni, mint amit a főmenüből az "MI/Játékszkript beállítások" alatt tudunk megnyitni.

\Section{Hasonló implementációk}

Az OpenTTD játék kiegészítők nélküli verziójában nem található előre beépített gépi ellenfél, minden hasonló implementációt a játék közössége hozott létre. Ezek mind rendelkeznek a játék fórumán egy bejegyzéssel, valamint elérhetőek és letölthetőek a játékon belül, a fórumbejegyzésben megadott linkről, vagy pedig az ún. “BaNaNaS” repository-ból, ahol egyébként az összes a játékban elérhető szkriptek, fejlesztések, köztük a gépi ellenfelek listája megtalálható, mellette linkelve az adott tartalom fórumbejegyzését. A meglévő AI-ok közül sok már nincs aktív fejlesztés alatt. A legtöbb fejlesztése a 2000-es évek végén kezdődött, aztán a 2010-es évek végére be is fejeződött. 2008 és 2017 között évente rendeztek AI bajnokságot, ahol különböző körökben egymásnak vetették az indulókat, és teljesítmény alapján értekelték azokat.

A különböző AI-ok mind különböző céllal jöttek létre. Vannak, amelyek egyes szállítás módszer lehető legjobb megvalósítására fókuszálnak, például TrAIns(Vonatokra), RoadAI(közúti járművekre), Chopper(Légi közlekedésre). Ezek mellett vannak általánosak is, például az AdmiralAI amelynek fókusza a lehető legtöbb funkció implementálása volt vagy a SimpleAI ami szintén általános és több szállítási módszert is alkalmazó AI. Vannak olyan AI-ok is amik lehet hogy nem a leghatékonyabban működnek, de van egy olyan tulajdonságuk ami érdekessé tenné azt, például a PathZilla ami gráfelmélet segítségével létrehozott úthálózatok építésére fókuszál, valamint többfajta útvonalkereső algoritmust alkalmaz.

Az AdmiralAI célja a játékhoz készített API legtöbb funkciójának kihasználása, valamint a változatos játékélmény biztosítása. Az utóbbit azzal próbálja meg elérni, hogy az AI minden szállítási módszert igyekszik alkalmazni. A mi szemszögünkből érdemes az AdmiralAI-al foglalkozni, mivel sok példát találhatunk a program forráskódjában, az API függvényeinek megvalósítására, valamint az AI felépítésére. A fájlszerkezet nagyon jól kitalált és felépített, minden összefüggő függvény, parancs, el van tagolva megfelelő osztályokba, mappákba, bár a kód értelmezése előzetes ismeretek nélkül elég nehézkes a kódhoz tartozó kommentekkel együtt is. Az AI sok inspirációt tud adni a sajátunk fejlesztéséhez, de előttes ismeretekre van szükségünk az API-ról a megfelelő szintű megértéshez.

A PathZilla már sokkal egy érdekesebb megvalósítása egy AI-nak, mivel gráfelméleti algoritmusokat használ a városok összekötésére, és útvonalak megtervezésére. Ezt Delaunay-háromszögelés módszerével valósítja meg, ahol a pontok halmaza a városok pozícióját jelenti a játék térképén.

Algoritmus/háromszögelés módjának részletesebb bemutatása??

A TrAIns egy 2009-ben fejlesztett intelligencia, ami egy brazil egyetem két hallgatója által készült el, mint kutatói munka. A céljuk az volt a projekttel, hogy egy olyan AI-t készítsenek, ami hatékonyan tudjon vasútvonalakat építeni és használni, ami akkoriban nem volt gyakori az AI-ok között, vagy nem hatékony megvalósítással rendelkezett. A korábbiakkal szemben, amelyek csak közvetlen összekötöttek állomásokat és egy vonat használta a pályát oda-vissza, ez az AI úgy épített vasútvonalakat, mint ahogyan azt egy játékos is tenné. Több vágányos vonalakat is épített, így lehetővé tette hogy egy vonalon egyszerre több vonat is közlekedhessen, és ezek a vonalak keresztezni is tudják egymást. Ennek a segítségével képes egy célba több egymáshoz közeli forrásból árut elszállítani úgy, hogy a vonal nagy részét több vonat is használja egyszerre.

\Section{Saját stratégiák}

Az eddig megemlített AI-ok nagyrésze mint vagy tudományosabb szempontból közelíti meg a játék problémáját. Ezzel szemben a saját megvalósításnál az én szempontom egy olyan AI készítése amely többnyire a saját játékstílusom megvalósítása a számítógép erejének segítségével. Ezalatt azt értem, hogy a probléma megközelítése emberi szempontból történik, azonban a tényleges döntéshozásokat már az AI fogja megtenni.

Amíg például a PathZilla esetében az útvonalak előre el vannak döntve a hármszögeléses módszer segítségével és az útvonalak később kerülnek megépítsére, az én játékos szemszögemből csak néhány arra érdemes város kerül összekötésre. Ezek általában a legnagyobb lakossággal rendelkező városok lesznek. Játék közben egy emberi felhasználó mire megvizsgálja a pálya jelenlegi helyzetetét, és meghatározza mondjuk két szomszédos városról hogy oda érdemes lehet egy buszvonalat létrehozni, ha az AI-nak meg tudjuk adni milyen szempontok alapján vizsgálja meg a pályát, lehet hogy ugyan arra, a döntésre, vagy jobbra tud jutni mint a felhasználó rövidebb idő alatt.