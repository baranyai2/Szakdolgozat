\Chapter{Bevezetés}

A dolgozat egy, az OpenTTD nevű vállalat szimulációs játékhoz készített mesterséges intelligencia (\textit{AI, Artificial Intelligence}) tervezését, implementációját és hatékonyságának vizsgálatát mutatja be.

A dolgozatban először az OpenTTD játékról kapunk egy nagyvonalú áttekintést. Maga a játék kifejezetten sokrétű, ezért itt éppen csak a legszükségesebb, az AI elkészítése szempontjából releváns részek kerülnek szóba.

A játék mesterséges intelligenciájának elkészítéséhez a Squirrel nevű programozási nyelv került felhasználásra. Ezt a nyelvet az OpenTTD esetében adottnak tekinthetjük, ugyanis a játék fejlesztői ezt a nyelvet építették bele a szoftverbe azért, hogy ahhoz különféle kiegészítőket (például esetünkben a mesterséges intelligenciát) hozzá lehessen adni a beépített funkciókhoz.

A saját készítésű mesterséges intelligencia bemutatása előtt azt láthatjuk, hogy a játékhoz példaként készített NoAI nevű intelligencia milyen interfésszel, implementálandó részekkel rendelkezik. Ezt követően röviden áttekintésre kerülnek az elterjedt, összetettebb, játékra alkalmas intelligencia implementációk.

% A saját AI-ról néhány áttekintő jellegű gondolat.

A dolgozat végén az elkészült intelligencia tesztfuttatásainak eredményeit, az azokkal kapcsolatos összehasonlításokat, észrevételeket láthatjuk.
