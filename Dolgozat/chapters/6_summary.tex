\Chapter{Összefoglalás}

A dolgozat a mesterséges intelligenciával kapcsolatos újdonságok mellett egy speciálisnak mondható, tipikusan játékok esetében előforduló programozási nyelvet is bemutatott.

A fejlesztést nagyban segítette az OpenTTD-hez rendelkezésre álló példa implementációk és az AI fejlesztéshez kialakított eszközkészlet.

A saját készítésű mesterséges intelligencia éppen csak egy kis szeletére koncentrált a játéknak. Az utak építése és a buszokkal való szállítás mellett érdemes lehet további kutatási irányként a vasúti szállítmányozás, a vízi és a légi közlekedés elemeire is kitérni. Szintén egy további, egyelőre ki nem használt lehetőséget ad a gyárak közötti szállítás megvalósítása a mesterséges intelligencia számára.

A megvalósított DebacleAI tesztfuttatásai és teszteredményei alapján nem teljesen versenyképes a korábbi, huzamosabb fejlesztési idővel rendelkező, részletesebben megvalósított AI-okkal szemben. Ezek sokkal kifinomultabban és gyorsabban képesek a DebacleAI által érintett problémákat kezelni. Azonban egy játékos számára, mint ellenfél bizonyos kihívást tud jelenteni, ezzel is színesítve az alap esetben egyjátékos módú játékélményt.

Az optimalizáláshoz elsősorban a várható utasszám és a városok közötti távolság adta a kiindulópontot. Ezeket normalizálva, egy jósági érték számításba építve kaptunk egy olyan plusz paramétert ($w_a$), amely hatását a későbbi, paraméterhangolással foglalkozó szakaszban becsülni lehetett. Szintén további fejlesztési lehetőségként adódik az ehhez hasonló optimalizálási problémák felírása, megoldása, ezek összehasonlítása különböző paraméterezések mellett.

Ahhoz, hogy ha valaki ténylegesen meg akarja érteni, hogy honnan jönnek a dolgozatban említett problémák, akkor nyilván célszerű a játékot kipróbálni, mindenképpen érdemes azt minél alaposabban megismerni.
Bízok benne, hogy a dolgozat egyaránt felkeltette az érdeklődést az OpenTTD játék, a mesterséges intelligencia fejlesztés, az optimalizálási problémák megoldása, esetlegesen a Squirrel nyelven történő programozás irányába.
